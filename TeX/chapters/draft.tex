\chapter{Drafting}
\nocite{*}
\begin{figure}[h]
    \centering
    \includesvg[width=.8\textwidth, inkscapelatex=true]{graphics/diagram}
    \caption[Sketch of the optical setup]{Sketch of the optical setup.}
    \label{fig:optical setup sketch}
\end{figure}

\begin{itemize}
    \item Laser: \url{https://www.thorlabs.com/thorproduct.cfm?partnumber=DJ532-10}
    \item Attenuator(s): \url{https://www.thorlabs.com/thorproduct.cfm?partnumber=ATT30/M-YAG}
    \item Adjustable Attenuator: \url{https://www.thorlabs.com/thorproduct.cfm?partnumber=NDC-100C-4M-A}
    \item Detector: \url{https://www.thorlabs.com/thorproduct.cfm?partnumber=SPDMA}
\end{itemize}

The photon count per second with given laser power \(P\) and wavelength \(\lambda\) expands as follows
\begin{align}
    f_{0,Ph} = \frac{P}{E_{Ph}} = P \cdot \frac{\lambda}{c h}
    \label{eq:unattenuated photons per second}
\end{align}

Since even the weakest laser source is still far too powerfull, hence emitting too much photons per second, an array of attenuators are used.
The proposed attenuators exhibit static behavior with a typical attenuation ratio of \qty{0.05}{\percent}.
To accomodate expected fluctuations in beam attenuation due to thermal effects, stray photons (what else?) an  attenuator with lower but adjustable attenuation coefficient is placed after the array.
\begin{align}
    f_{n,Ph} = f_{0,Ph} \cdot \mu^n
    \label{eq:attenuated photons per second}
\end{align}
where \(n\) is the number of static attenuating elements.

\section{Attenuation}
    To detect and distinguish single (or pseudo-single) photons it is mandatory to reduce the primary source power by many orders of magnitude until the (average) photon flux at the active detector area (\qty{500}{\micro\metre}) reaches a level well below its dead-time.
    \begin{align}
        I_n = \frac{P_n}{A} &= f_{0,Ph} \cdot \mu^n \cdot \frac{c h}{\lambda}\\
        &\Leftrightarrow\\
        N_{detector} &=
    \end{align}
\section{Interesting Links}
    \url{https://www.sps.ch/artikel/progresses/wave-particle-duality-of-light-for-the-classroom-13/}

    \url{http://marty-green.blogspot.com/2016/02/there-are-no-pea-shooters-for-photons.html}
    
    \url{https://www.rp-photonics.com/spotlight_2015_02_05.html}